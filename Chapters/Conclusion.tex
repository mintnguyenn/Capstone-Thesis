\chapter{Conclusion}
In conclusion, this capstone report proposes an approach for camera calibration of multiple partially overlapping cameras to determine the extrinsic parameters of the camera system. This project is a part of the "Objective real-time live cattle assessments to improve profit" funded project by Meat and Livestock Australia.

As a part of camera calibration, this report proposes to use an augmented planar board as the calibration target. A double-sided planar board with ArUco markers is employed to solve the problem with cameras facing each other and having overlapping fields of view. In addition, the General Graph Optimisation (G2O) is applied to optimise the calibration results. Experiments were also implemented to test the approach's feasibility, and the results obtained were also very encouraging. However, this method still needs improvement to increase efficiency and accuracy.

In the future, the program framework, as well as the algorithm, would be improved to test with the whole 16 cameras of this system. We would intend to publish our approach as a ROS package and store it on a public repository on GitHub for further development. 

